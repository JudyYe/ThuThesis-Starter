\chapter{引言}
 
\section{OCT图像病态定位研究意义} % (fold)
    光学相干成像(Optical Coherence Tomography, OCT) 。它是一种利用光的弱相干干涉扫描得到的成像技术。从上世纪九十年代开始,被逐渐应用于医学图像分析,尤其是视网膜的成像。由于它有微米级的成像分辨率,且可以由此生成三维模型,所以光学相干成像技术已经被应用到眼科的临床问题,尤其是例如视网膜黄斑区这种区域面积小,厚度薄的部位。视网膜上的黄斑区大小只有2毫米左右,厚度只有0.1毫米。它与人的视力密切相关,是人眼视觉最敏锐的地方。这个区域常见的病症包括老年黄斑变性,感性黄斑水肿,糖尿病性黄斑水肿、黄斑裂孔等。医生在诊断黄斑区附近病变时,经常借助OCT观察病变位置,做出诊断。

    然而,由于OCT成像技术分辨率很高,一位患者主体得到的扫描图像数量大,靠人工搜索寻找病态结构是一件费时费力的工作。并且,该项工作需要较强的专家知识,对人工识别技术也有较高要求。因此,如何利用计算机图像处理的方法,识别出病态特征,并准确分类,将有很高的临床应用价值。
    
    在计算机辅助医疗的实际应用中,医生不仅需要知道系统分类的输出结果,由于医疗行业的特殊性和复杂性,人们更希望知道系统做出该推断的原因。这也是问答系统在计算医疗领域广受重视的原因。随着眼底黄斑区的OCT图像分类技术在公开数据集上取得较好的成果,人们不仅希望得到一个分类结果,更希望知道系统做出分类的依据。在视网膜分类的场景中,如果系统还输出它推断的可能病变位置,这将极大地帮助医生有效利用计算机信息作出诊断,对计算机辅助医疗的进一步推广有促进作用。

    据笔者所知,关于眼底区域OCT图像的现有研究主要集中在图像分割、图像分类等方向,而在病变定位的方面的研究较少。主要原因在于:研究较多的眼底OCT分割任务大多是围绕提取底层视觉特征为主的区域分层,比如基于边、邻近区域等,相对而言,病变位置是含有更多语义信息的概念,因而相应的特征表示要求更加抽象;相较于分类,病变定位需要输出跟空间位置相关的信息,难度更高。同时还要面对数据量少,人工标注成本大等问题。本文提出了一种创新的算法,试图解决这些问题,并在两个数据集上验证评估了算法的效果。

\section{本文章节结构} % (fold)
    本文的组织遵从以下结构。第二章将更详细地描述定义问题,并对与该问题相关的工作做文献综述。第三章描述了本文提出的算法,从一个简单的数学动机出发,对图像特征表示进行设计。具体的,预设一些性质,选取适当的数学工具,使提取的特征向量满足预设性质,进而构建一个分类与病态定位的预测系统。在第四章里,实验验证了第三章提出的算法,从视觉评估和定量评估两个层面分析算法提取的特征向量是否满足预设性质,该章结尾深入模型,试图给出对模型习得知识的理解。在第五章中,笔者回顾了该篇工作的整体思路,提出了现有的问题和改进的方向。

