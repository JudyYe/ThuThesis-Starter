\chapter{概述}
    本章将首先定义解决的问题,给出形式化表达,指出面对的挑战。然后,将综合论述OCT眼底黄斑区图像领域、图像定位方面的相关工作研究,并详细讨论了本篇文章的关键组件有监督字典学习的分类技术。

\section{问题描述}
    本问题的最终目的是协助医疗工作者诊断眼底黄斑区的OCT图像,诊断包括两方面:一,对一张输入图片,输出一个类别;二,输出系统推测为该类别的可能原因,即病态定位。

    然而,由于医学诊断需要专业的医疗知识,所以病变位置信息的人工标注的难度和成本都会很高,据笔者所知,目前并没有带有病态位置标注信息的公开数据集,而以本项目的有限资源,并不足以自主构建一个带有位置信息标注的足够打的数据集。另一方面,医学图像与自然图像不同,它没有客观的标准明确的边界。比如,自然图像中,狗的轮廓是清晰的,人工很容易将狗的图像与其它物体分离,但由于病变通常是缓慢平缓的,病变部位与正常部位难以看到明确的分界,即使专业医生标注,也仍然会因为主观判断这些模糊边界而使标注产生误差。因此,本文工作将在没有任何病态位置信息标注的情况下,试图输出一个带有位置信息的病变位置推断,监督信号只有图像级别的类别标签。

    缺少标注,也为评估算法的工作产生了难度。本文遵循两个原则设计评估方案:1. 尽量寻找定量的标准衡量算法性能,比如,分类准确率,与原图像的重构误差等。2. 当因缺乏真实标注而无法进行定量评估的时候,采用视觉评估的手段,但注意对于得到图像的理解、对模型的解释。由于篇幅所限,不能展示所有的视觉结果,笔者选取尽量典型的例子,既有效果较好的例子,又又效果较差的例子。更多的实验结果将公布到网站。




\section{相关工作}

    \subsection{OCT眼底黄斑区图像的技术现状}
    \subsection{图像定位技术}
    \subsection{基于有监督字典学习的分类技术}

\section{创新点}