\chapter{概述}
    本章将首先定义解决的问题,给出形式化表达,指出面对的挑战。然后,将综合论述OCT眼底黄斑区图像领域、图像定位方面的相关工作研究,并详细讨论了本篇文章的关键组件有监督字典学习的分类技术。

\section{问题描述}
    本问题的最终目的是协助医疗工作者诊断眼底黄斑区的OCT图像,诊断包括两方面:一,对一张输入图片,输出一个类别;二,输出系统推测为该类别的可能原因,即病态定位。

    然而,由于医学诊断需要专业的医疗知识,所以病变位置信息的人工标注的难度和成本都会很高,据笔者所知,目前并没有带有病态位置标注信息的公开数据集,而以本项目的有限资源,并不足以自主构建一个带有位置信息标注的足够大的数据集。另一方面,医学图像与自然图像不同,它没有客观的标准明确的边界。比如,自然图像中,狗的轮廓是清晰的,人工很容易将狗的图像与其它物体分离,但由于病变通常是缓慢平缓的,病变部位与正常部位难以看到明确的分界,即使专业医生标注,也仍然会因为主观判断这些模糊边界而使标注产生误差。因此,本文工作将在没有任何病态位置信息标注的情况下,试图输出一个带有位置信息的病变位置推断,监督信号只有图像级别的类别标签。

    缺少标注,也为评估算法的工作产生了难度。本文遵循两个原则设计评估方案:1. 尽量寻找定量的标准衡量算法性能,比如,分类准确率,与原图像的重构误差等。2. 当因缺乏真实标注而无法进行定量评估的时候,采用视觉评估的手段,但注意对于得到图像的理解、对模型的解释。由于篇幅所限,不能展示所有的视觉结果,笔者选取尽量典型的例子,既有效果较好的例子,又有效果较差的例子。

    因此,本文要面对的问题是,在眼底黄斑区OCT成像的场景下,有一个只有图像级粒度标注的数据集,即类别标注$C$的数据集$X$,在其中训练一个模型$M=(\varphi, \phi )$,使得给定一个输入$x$,输出一个类别标签$\tilde{c} = \varphi(x)$ 和关于$x$的病变位置信息$\phi(x) \in \mathbb{R} ^{width \times height}$。使得

    \begin{align}
        \min _{\varphi, \phi} \left(\mathcal{L}(c, \varphi (x)) + \lambda f(\phi(x)) \right)
    \end{align}

    其中$f$是一个未知的评价函数,评估推测的病态位置信息。$\mathcal{L}$是分类误差。相对$\varphi, \phi$,后者的训练更加困难。本文将给出训练$\varphi, \phi$的方法,在这之前,还需要确定在没有真是标注下$f$的定义。第三章提出了一种替代性的方法去近似该优化问题。

\section{相关工作}

    \subsection{OCT眼底黄斑区图像的技术现状}
    过去二十年,研究主要围绕着视网膜层分割算法\cite{antony2013combined,savastano2014differential},先分割出OCT图像中明亮的视网膜层,然后预处理图片,统计分割的层厚度,与数据库中的测量厚度对比,从而识别病变的位置。然而,当病变加深的时候,视网膜显著的层结构不再明晰,很难再对视网膜分层。另外的局限在于,无法推广到多分类任务。因此,如何对OCT图像进行直接自动分类受到更多的研究。现有的方法框架主要分为三部分:预处理,图像表示与分类。

    \textbf{图像预处理}
    较小样本的训练数据要训练较强学习能力的模型非常容易过拟合的问题\cite{srivastava2014dropout}。充分考虑OCT的高度结构化的特征,同人脸分类类似,对齐图像可以提高识别率\cite{cao2014face}。我们希望自动预处理所有眼底视网膜区OCT图像。文献\inlinecite{srinivasan2014fully} 使用了先对视网膜色素薄层(Retinal Pigment Epithelium, RPE)进行分割,找到RPE层后,拉直该层,对齐图像。然而,当病变严重时,难以分割该层,所以,文献\inlinecite{liu2011automated}中没有使用分割算法,然而却无法处理视网膜肿胀的病例。文献\inlinecite{sun2017fully}改进了文献\inlinecite{srinivasan2014fully}中的方法,使用多种拟合方法,处理更多的病例情况,在现有数据集上,该对齐方法的适应性最强,可以被直接应用到我们的研究中。

    \textbf{图像表示}
    很早OCT图像中的降维问题就引起了讨论,如文献\inlinecite{liu2011automated}使用主成分分析降维,结合基于多尺度空间金字塔和局部二分模式(Local binary pattern, LBP)直方图学习的方法表征图像。2012年,基于图的表示方法被证明在OCT图像分类表示上取得了较好的效果\cite{zheng2012automated,hijazi2012data},它考虑了空面位置和结构信息。首先,一张图被分解为四叉树集,然后,子图挖掘技术被用于生成的四叉树以寻找通用的子图来生成全局向量来表示每张图像。之后,忽略削弱位置信息的表示方法不断取得更好的分类效果\cite{srinivasan2014fully}。用直方图梯度统计(Histogram of Oriented Gradients, HOG)来表示每个图像.近些年,稀疏表示在模式识别和图像处理上的成功,促使医学图像也引入稀疏假设来处理医学影像,\cite{oliveira2014medical,wang2015predict,afzali2016medical}.比如文献\inlinecite{sun2017fully}使用稀疏编码尺度不变描述子SIFT(Scale Invariance Feature Transformation) 结合空间金字塔表示图像。但是尺度不变描述子表示失去了图像的空间位置信息,虽然在分类任务上非常有效,却不能应用在本文的问题中。

    \textbf{图像分类与稀疏编码}
    在当前的图像处理领域,图像稀疏编码(sparse coding)与字典学习(dictionary learning)不断在模式识别(pattern recognition)机器学习(machine learning),计算机视觉(computer vision)和医学视觉(medical vision)上取得新进展。在人脸识别\cite{zhang2010discriminative,wright2009robust},场景分类\cite{lazebnik2006beyond,gao2010kernel}等领域,运用判别式字典(discriminative dictionary),流行学习(manifold learning),稀疏编码等方法,不断取得新的效果。根据观察,除了普遍物体分类,医学OCT图像分析还可以参考人脸检测的方法,因为OCT高度结构化的特征。

    在稀疏编码中的假设是,高维度的图片分布在稀疏的低维空间中,并且在低维空间中,有较好的线性性质。\cite{huang2006sparse}人们认为,应用稀疏编码可以得到更代表本质特征的表示。一类图像信号的稀疏表示,可以表示成一些基本元素的集合,这些元素构成了过完备的字典(over-complete),而一类图像的稀疏系数,只需要字典中的一些元素就可以表示。在字典学习中,有两大类方法:预定义的字典和数据驱动的字典。前者的字典元素被提前定义好的,如小波集合,傅里叶集合等;后者的字典元素是从训练数据中学习得到的。



    \subsection{弱监督的图像分割技术}
    图像分割技术在深度学习之前,一直没有很好的方法输出高密度的分数匹配(dense score map)。经典方法通常把问题划归到视觉显著性(visual saliency)\cite{han2006unsupervised,donoser2009saliency,yang2008unsupervised,chang2011co}等问题,自下而上地解决问题,但是这种缺少监督信号得到的方法,只能得到较低层信息的分割结果,比如分割根据梯度、颜色等依据而组成的小区域。在自上而下的监督学习中,最近随着全卷积神经网络\cite{long2015fully}的出现,它大大提高了分割性能,给密集的分数匹配输出这一类方法提供了很好的范本。但是,由于获取分割位置信息比获取类别标签的成本高很多,在大规模数据集的应用中,几乎很难获取有像素级别信息标注的数据。因此,弱监督下的图像分割技术也引起了很多关注。弱监督是指,训练数据的监督信号不含像素级粒度的标注,这类问题在近两年取得了较好的进展,但大多都在深度学习领域\cite{hong2015decoupled,pathak2014fully,oquab2015object,wei2016stc,papandreou2015weakly}。

    然而,深度学习需要百兆级别大数据集\cite{deng2009imagenet}或迁移学习\cite{shin2016deep,wangtransfer}来防止过拟合。尽管文献\inlinecite{tajbakhsh2016convolutional,schlegl2014unsupervised,carneiro2015unregistered}表示在医学影像中,在大规模自然图像上做预训练的模型可以迁移到医学图像数据集上。但是,本工作拥有的数据集仍然小于其文章中的医学影像训练数据集。因此,虽然深度学习在近几年弱监督图像分割领域取得了较好的进展,该方法也无法应用于本次任务。只能从非深度学习的“经典”方法中,设计算法完成任务。





    \subsection{基于有监督字典学习的分类技术}
    字典学习和稀疏编码的动机在于最小化重建误差,并最早应用于图像重建,图像降噪等方向,取得了较好的效果。同时,字典学习与稀疏编码近来也发现在分类问题上可以取得较好的效果。但是,在分类任务中的目标,则转变成,学习有区分度的字典和对应的稀疏表示使得不同类别之间有尽量大的区分度。因而产生了“判别式字典学习”(discriminative dictionary learning)的学习技术,在学习时引入类别标记(label)。我们称之为,有监督的字典学习和稀疏编码(S-DLSR, supervised dictionary learning and sparse representation),以提升稀疏编码在分类任务中的效果。我们根据文献\inlinecite{gangeh2015supervised}中的分类方法,将有监督的字典学习和稀疏编码技术分为以下六类: 1)每一类学习一个字典\cite{wright2009robust,yang2010metaface}; 2)无监督学习字典后做有监督的调整\cite{fulkerson2008localizing} ;3)联合学习字典与分类器\cite{mairal2008discriminative,jiang2013label,zhang2010discriminative,pham2008joint,yang2008unifying};4)嵌入类别标签的字典学习\cite{yang2011fisher,rodriguez2008sparse};5)嵌入稀疏表示的字典学习 \cite{gangeh2013kernelized,zhang2013simultaneous,lazebnik2009supervised};6)字典元素直方图的学习。\cite{lian2010probabilistic,zhang2009learning}
    它们的区别主要在于判别字典的原子是否有强类别标签。本文采用的方法是第三类\cite{jiang2013label}的方法,它的主要优点在于,共同学习分类器和字典,提升分类准确率,另外,每一个原子都有一个类别标签,可以根据这个特性来“操纵”特征表示,该方法在人脸识别任务上取得了较好方法。后面的分析我们将发现,人脸识别跟OCT图像有很多相似的地方。

\section{创新点}
    本项工作的贡献有如下几方面:
    \begin{enumerate}
        \item 对只有类别标签却输出位置信息的问题,给出了不使用深度学习的一种可以适用于中等规模数据集的方法。
        \item 对于因为缺少真实标注而定量评价算法性能的问题,给出了一种合理的替代方法。
        \item 这是在眼底黄斑区OCT成像领域,较少的一篇关于病态定位问题的研究。
    \end{enumerate}
